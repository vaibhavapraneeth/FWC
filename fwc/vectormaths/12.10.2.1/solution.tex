\documentclass[12pt]{article}
\usepackage{graphicx}
%\documentclass[journal,12pt,twocolumn]{IEEEtran}
\usepackage[none]{hyphenat}
\usepackage{graphicx}
\usepackage{listings}
\usepackage[english]{babel}
\usepackage{graphicx}
\usepackage{caption} 
\usepackage{hyperref}
\usepackage{booktabs}
\usepackage{commath}
\usepackage{gensymb}
\usepackage{array}
\usepackage{amsmath}   % for having text in math mode
\usepackage{listings}
\lstset{
  frame=single,
  breaklines=true
}
  
%Following 2 lines were added to remove the blank page at the beginning
\usepackage{atbegshi}% http://ctan.org/pkg/atbegshi
\AtBeginDocument{\AtBeginShipoutNext{\AtBeginShipoutDiscard}}
%


%New macro definitions
\newcommand{\mydet}[1]{\ensuremath{\begin{vmatrix}#1\end{vmatrix}}}
\providecommand{\brak}[1]{\ensuremath{\left(#1\right)}}
\providecommand{\norm}[1]{\left\lVert#1\right\rVert}
\newcommand{\solution}{\noindent \textbf{Solution: }}
\newcommand{\myvec}[1]{\ensuremath{\begin{pmatrix}#1\end{pmatrix}}}
\let\vec\mathbf


\begin{document}
\begin{center}
\title{\textbf{Vectors}}
\date{\vspace{-5ex}} %Not to print date automatically
\maketitle
\end{center}
\setcounter{page}{1}
\section*{12$^{th}$ Maths - Exercise 10.2.1}

\begin{enumerate}
\item Compute the magnitude of the following vectors 
$\\ \overrightarrow{a}=\hat{i}+\hat{j}+\hat{k},\overrightarrow{b}=2\hat{i}-7\hat{j}+3\hat{k}$ and $\overrightarrow{c}=\dfrac{1}{\sqrt{3}}\hat{i}+\dfrac{1}{\sqrt{3}}\hat{j}-\dfrac{1}{\sqrt{3}}\hat{k}$.\\
\solution
\begin{align}
\text{Let } \vec{a} = \myvec{1\\1\\1} , \vec{b} = \myvec{2\\ -7 \\ 3},\vec{c} = \myvec{\dfrac{1}{\sqrt{3}}\\ \dfrac{1}{\sqrt{3}} \\ -\dfrac{1}{\sqrt{3}}} 
\label{eq:1}
\end{align}
let us assume magnitudes of $\vec{a},\vec{b},\vec{c}$ are $\norm{\vec{a}},\norm{\vec{b}},\norm{\vec{c}}$ respectively
so
\begin{align}
	\norm{\vec{a}}&={\vec{a}}^{\top}\vec{a}, 
	\label{eq:3}
	\\ \norm{\vec{b}}&={\vec{b}}^{\top}\vec{b}, 
	\label{eq:4}
	\\ \norm{\vec{c}}&={\vec{c}}^{\top}\vec{c}	
	\label{eq:5}
\end{align}

now substituting values of \eqref{eq:1} in \eqref{eq:3},\eqref{eq:4} and \eqref{eq:5} respectively we get

the magnitudes of $\norm{\vec{a}}=\sqrt{3}$, $\norm{\vec{b}}= \sqrt{62}$, $\norm{\vec{c}}=1$

\end{enumerate}
\end{document}