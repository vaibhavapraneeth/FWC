\documentclass[12pt]{article}
\usepackage{graphicx}
\usepackage[none]{hyphenat}
\usepackage{graphicx}
\usepackage{listings}
\usepackage[english]{babel}
\usepackage{graphicx}
\usepackage{caption} 
\usepackage{booktabs}
\usepackage{array}
\usepackage{amssymb} % for \because
\usepackage{amsmath}   % for having text in math mode
\usepackage{extarrows} % for Row operations arrows
\usepackage{listings}
\usepackage[utf8]{inputenc}
\lstset{
  frame=single,
  breaklines=true
}
\usepackage{hyperref}
  
%Following 2 lines were added to remove the blank page at the beginning
\usepackage{atbegshi}% http://ctan.org/pkg/atbegshi
\AtBeginDocument{\AtBeginShipoutNext{\AtBeginShipoutDiscard}}


%New macro definitions
\newcommand{\mydet}[1]{\ensuremath{\begin{vmatrix}#1\end{vmatrix}}}
\providecommand{\brak}[1]{\ensuremath{\left(#1\right)}}
\newcommand{\solution}{\noindent \textbf{Solution: }}
\newcommand{\myvec}[1]{\ensuremath{\begin{pmatrix}#1\end{pmatrix}}}
\providecommand{\norm}[1]{\left\lVert#1\right\rVert}
\providecommand{\abs}[1]{\left\vert#1\right\vert}
\let\vec\mathbf
\begin{document}
\begin{center}
\title{\textbf{  Vectors}}
\date{\vspace{-5ex}} %Not to print date automatically
\maketitle
\end{center}
\setcounter{page}{1}\section{12$^{th}$ Maths - Chapter 10}
\textbf{This is Problem-6 from Exercise 10.5 (Miscellaneous Exercise)}
\begin{enumerate}

\item Find a vector of magnitude 5 units, and parallel to the resultant of the vectors $\vec{a} =\myvec{2\\3\\-1}$ and $\vec{b}=\myvec{1\\-2\\1}$
\section{Solution}
Let us assume that required vector as $\vec{x}$\\ 
Given $\vec{a}$ and $\vec{b}$ are
\begin{align}
\vec{a}&=\myvec{2\\3\\-1}\\
\vec{b}&=\myvec{1\\-2\\1}
\end{align}
now assume that
\begin{align}
\vec{a}+\vec{b}&= \vec{c}\\
\vec{c}&= \myvec{2\\3\\-1}+\myvec{1\\-2\\1}\\
\vec{c}&=\myvec{3\\1\\0}
\end{align}
so our resultant vector is $\vec{c}=\myvec{3\\ 1\\ 0}$\\
\begin{align}
 \frac{\vec{c}}{\norm{\vec{c}}} &= \frac{1}{\sqrt{10}}\times\myvec{3\\ 1\\ 0}\label{6}
\end{align}
so \eqref{6} is a unit vector which is in the  direction of resultant vector of $\vec{a}$ and $\vec{b}$ but we want the $\vec{x}$ which  magnitude is  5 so 
\begin{align}
 \vec{x}&=5\times \frac{1}{\sqrt{10}}\times\myvec{3\\ 1\\ 0} \\
 \vec{x}&=\myvec{\frac{3\sqrt{10}}{2}\\ \frac{\sqrt{10}}{2}}
\end{align}
Hence the required vector $\vec{x}$ is $\myvec{\frac{3\sqrt{10}}{2}\\ \frac{\sqrt{10}}{2}}$
\end{enumerate} 
\end{document}